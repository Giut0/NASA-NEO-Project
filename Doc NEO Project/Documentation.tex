\documentclass[italian,12pt,a4paper]{article}
\usepackage[utf8]{inputenc}
\usepackage[T1]{fontenc}
\usepackage{babel}
\usepackage{graphicx}
\usepackage{hyperref}
\title{Università degli studi di Bari facoltà di scienze MM.FF.NN}
\date{} % clear date
\hypersetup{
	colorlinks=true,
	linkcolor=black,
	filecolor=magenta,      
	urlcolor=cyan,
	pdfpagemode=FullScreen,
}
\graphicspath{ {./images/} }
\usepackage{tocloft}


\cftsetindents{section}{0em}{2em}
\cftsetindents{subsection}{0em}{2em}

\renewcommand\cfttoctitlefont{\hfill\Large\bfseries}
\renewcommand\cftaftertoctitle{\hfill\mbox{}}

\setcounter{tocdepth}{2}
\begin{document}
	\maketitle
	\thispagestyle{empty}
	\begin{center}
		\huge	\textbf{Progetto Data Mining} \\
		\Large \textbf{NASA - Nearest Earth Objects hazard detection}
	\end{center}
	
	
	
	\begin{center}
		by \\
		\Large \textbf{Vito Proscia mat. 735975}
	\end{center}

	
	\begin{figure}[hb]
		\centering
		\includegraphics[width=5cm]{image.png}
	\end{figure}
	
	\vfill
	\begin{center}
		Anno accadenico 2022-2023
	\end{center}
	
	\newpage
	
	\tableofcontents

	
	\newpage

	
	\section{Introduzione}
	
	\subsection{Descrizione del dataset}
	
	
	\href{https://www.kaggle.com/datasets/sameepvani/nasa-nearest-earth-objects/}{Near-Earth Objects} (NEO) dataset contiene una serie di informazioni, raccolte dalla NASA, che caratterizzano degli oggetti rilevati vicino alla terra, molti di questi oggetti sono a migliaia di chilometri dalla superficie terrestre, ma su scala astronomica queste distanze sono molto piccole e possono influenzare fenomeni naturali, quali per es: ... \\
	La natura dei Near-Earth Objects (NEO) si può dividere in:
	\begin{itemize}
		\item \textbf{Comete}: corpo celeste relativamente piccolo, composto da gas ghiacciati frammenti di rocce e metalli
		\item \textbf{Asteroidi}: corpi minori di un sistema planetario originati dallo stesso processo di formazione dei pianeti ma le cui fasi di accrescimento si sono interrotte più o meno presto
	\end{itemize}
	

	\subsection{Analisi features}
	Il dataset inizialmente si compone di dieci features che vanno a descrivere una serie di caratteristiche dei corpi celesti registrati, in particolare abbiamo:
	
	\begin{enumerate}
		\item	\textit{id} [numeric]: identificatore univoco per ogno oggetto
		\item 	\textit{name} [string]: nominativo dato dalla NASA
		\item \textit{est\_diameter\_min} [numeric]: diametro minimo stimato (Km)
		\item \textit{est\_diameter\_max} [numeric]: diametro massimo stimato (Km)
		\item \textit{relative\_velocity} [numeric]: Velocità relativa rispetto alla terra (Km/h)
		\item \textit{miss\_distance }[numeric]: ???
		\item \textit{orbiting\_body} [string]: Corpo rispetto al quale l’oggetto sta orbitando
		\item \textit{sentry\_object} [boolean]: Copro incluso o meno in sentry (sistema di monitoraggio automatico delle collisioni)
		\item \textit{absolute\_magnitude} [numeric]: descrizione della luminosità dell’oggetto (energia radiata dal corpo al secondo)
		\item \textit{hazardous} [boolean]: Indica se il corpo è pericoloso o meno
	
		
	\end{enumerate}
	
	

	

	
\end{document}